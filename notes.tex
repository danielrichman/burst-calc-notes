\documentclass{article}
\usepackage[a4paper,left=2cm,top=2cm]{geometry}

\usepackage{parskip}
\usepackage{amsmath}
\usepackage{amssymb}
\usepackage{mathtools}
\usepackage{hyperref}

\begin{document}

\title{Burst Calculator}

There are three key models/assumptions/components in the busrt calculator:

\begin{itemize}
    \item A simple model, that works at sea-level (only), gives us a
          relation between (mass of helium) and (sea level ascent rate).
    \item The ascent rate is approximately constant wrt. height
    \item A basic pressure model for the Earth, coupled with the `fact'
          that the pressure \& temperature inside \& outside the balloon are
          equal, giving a relationship between (mass of helium) and (burst altitude).
\end{itemize}

All of these assumptions are wrong, to varying degrees.

\setcounter{section}{-1}
\section{Letters}

\begin{align*}
\intertext{We know these things \& constants}
    m_B &= \text{mass of balloon} &
    m_P &= \text{mass of payload} \\
    \rho_{G,0} &= \text{density of balloon gas at sea level} &
    \rho_{A,0} &= \text{density of air at sea level } \\
    C_D &= \text{drag coefficient} &
    g &= 9.81 \\
    a_0 &= \text{initial altitude} &
    r_B &= \text{burst radius of balloon} \\
    P_0 &= \text{pressure at sea level} \\
%
\intertext{and are interested in these (related) things, some of which
           will be specified, and some will be calculated}
    V_0 &= \text{volume of balloon at launch} &
    a_B &= \text{burst altitude} \\
    v_0 &= \text{instantaneous ascent rate just after launch} &
    T &= \text{time until burst} \\
%
\intertext{we have these trivial relationships}
    V_0 &= \frac{4}{3} \pi r_0^3 &
    V_B &= \frac{4}{3} \pi r_B^3 \\
    m_G &= V_0 \rho_{G,0} \quad \text{mass of gas} &
    m &= m_B + m_P + m_G \\
    A_0 &= \pi r_0^2 \quad \text{cross-sectional area at launch} \\
%
\end{align*}

\section{Sea level ascent rate}

The balloon reaches equilibrium/terminal velocity almost instantly,
so we can calculate this by equating forces vertically.

\begin{align*}
    \text{Gravity} && -mg \\
    \text{Buoyancy} && \rho_{A,0} g V_0 \\
    \text{Drag} && -\frac{1}{2} \rho_{A,0} C_D A_0 v_0^2
\end{align*}

We can solve for $v_0$ in terms of $r_0$:

\[
    v_0
    = \sqrt{
        \frac{ \rho_{A,0}gV_0 - mg }%
             { \frac{1}{2} \rho_{A,0} C_D A_0 }
        }
%
    = \sqrt{
        \frac{ 2g ((\rho_{A,0} - \rho{G,0}) (\frac{4}{3} \pi r_0^3) - m_B - m_P) }%
             { \rho{A,0} C_D \pi r_0^2 }
        }
\]

We can also solve for $r_0$ in terms of $v_0$; we need to solve this cubic in $r_0$:

\[
       8 \pi g (\rho_{A,0} - \rho_{G,0}) r_0^3
     - 3 \pi \rho_{A,0} C_D v_0^2 r_0 ^2
     - 6 (m_B + m_P) g
      = 0
\]

Note that if you differentiate wrt. $r_0$ and look for turning points, you should find two:
one at $0$, and one at some $r_0 > 0$. Because that cubic is negative at $r = 0$, we know
that it has exactly one real root, which moreover is positive.

\section{Ascent rate is constant}

This is approximately correct. It's quite close to a straight line
(try plotting it; then try numerically differentiating and plotting that).

It's worth noting that the model described in the first section does not even remotely hold
above sea level; in fact it predicts that the ascent rate will be directly proportional
to $\sqrt{r}$ (where $r$ is the radius of the balloon). Since our balloons pop at over
$3$ times the launch radius, and the ascent rate is fairly constant,
this is not consistent with reality.

This paper \url{http://www.atmos-meas-tech.net/4/2235/2011/amt-4-2235-2011.pdf} presents
a substantially more advanced model (after introducing the above model and explaining
where it is wrong), however using their model in practice is incredibly computationally
intensive.

Anyway, assuming ascent rate is constant, we have

\[  T = (a_B - a_0) / v_0 \]

\section{Bursting}

We have a model for the Earth's atmosphere; a function $\rho(a)$ that gives
air pressure for a certain altitude.
TODO: details; is it just the US Standard Model?

We assume that the pressure and temperature inside the balloon is equal
to the pressure and temperature outside of the balloon.

The balloon gas, and the amount of air displaced by the balloon, both
obey the ideal gas law $PV=nRT$. By various assumptions, $\frac{PV}{RT}$ is
equal for both gases ($R$ is a constant).
Therefore, the number of moles of balloon gas $n_G$ is equal to the number
of moles of air displaed $n_A$. But $n_G$ is constant wrt. time and altitude,
so $n_A$ is constant wrt. altitude. In particular, the mass of air displaced
remains constant.

This relates volume $V$ at altitude $a$ to air density: 
\[
    \rho_{A,0} V_0 = \rho(a) V
\]

The balloon bursts when the radius is $r_B$, i.e., when its volume is $V_B$, so

\[
    \rho_{A,0} r_0^3 = \rho(a_B) r_B^3
\]

There's a discussion of how the temeperature outside and inside differs in
the paper linked above. It is however close enough (we think).
\end{document}
